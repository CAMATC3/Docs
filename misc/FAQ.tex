%Este trabalho está licenciado sob a Licença Creative Commons Atribuição-CompartilhaIgual 3.0 Não Adaptada. Para ver uma cópia desta licença, visite https://creativecommons.org/licenses/by-sa/3.0/ ou envie uma carta para Creative Commons, PO Box 1866, Mountain View, CA 94042, USA.


\documentclass[12pt]{report}
\usepackage[brazil]{babel}
\usepackage[utf8]{inputenc}
\usepackage[T1]{fontenc}
\usepackage{hyperref}


\begin{document}
\chapter*{Respostas a perguntas frequentes}
\begin{enumerate}
\item  O que é o REAMAT?
É um projeto de extensão registrado no Instituto de Matemática e Estatística da Universidade Federal de Rio Grande do Sul cujo objetivo é produzir recursos educacionais abertos (REA) de matemática através de um processo colaborativo envolvendo toda a comunidade.

\item O que são recursos educacionais abertos (REA)?

REA são materiais de ensino, aprendizado e pesquisa disponibilizados em qualquer suporte ou mídia, seja física ou digital licenciados de forma livre, isto é, que propiciem pelo menos as seguinte quatro liberdades:
\subitem i. Usar: Uso do material para quaisquer fins, inclusive comercial.

\subitem ii. Adaptar: Liberdade de adaptar e melhorar os REA para que melhor se adequem às suas necessidades.

\subitem iii. Recombinar: Liberdade de combinar e fazer misturas e colagens de REA com outros REA para a produção de novos materiais.

\subitem iv. Distribuir: Liberdade de fazer cópias e compartilhar o conteúdo original e a sua versão derivada ou combinada.

\item Posso produzir material próprio derivado do conteúdo REAMAT?
Sim, mas você deve respeitar a licença escolhida pelo REAMAT, a \href{https://creativecommons.org/licenses/by-sa/3.0/}{Creative Commons Atribuição-CompartilhaIgual 3.0 Não Adaptada (CC-BY-SA 3.0)}. Isto implica, entre outros, licenciar seu trabalho de forma livre. Assim, se você produzir obra derivada para comercializar, os leitores do livro terão o direito de copiá-la e distribuir cópias. Veja \href{https://www.ufrgs.br/reamat/participe.html}{como participar}.

\item Preciso de autorização para usar o material do REAMAT?
Não. No entanto, ficamos felizes de ser notificados quando alguém gostou do material e o está utilizando.

\item Posso editar o conteúdo se não sou professor ou se não sou da UFRGS?
Sim, qualquer um pode participar da escrita. Veja \href{https://www.ufrgs.br/reamat/participe.html}{como colaborar}.

\item Qualquer um pode alterar livremente o conteúdo a qualquer momento?
Não, todas as alterações devem ser aprovadas por um organizador do projeto. Veja lista de \href{https://www.ufrgs.br/reamat/organizadores.html}{organizadores}.

\item Posso melhorar o conteúdo do livro acrescentando material de outras fontes?
Depende, você deve respeitar os direitos de autor das fontes consultadas e só pode usar material cuja licença seja compatível com a do REAMAT. A maioria dos livros e materiais didáticos disponíveis não são livres e seu uso no REAMAT constitui violação de direitos autorais. Atenção: material disponível na internet não necessariamente é livre.

\item Possuo material que produzi para uso ao longo de meu trabalho como professor. Posso acrescentar aos conteúdos do REAMAT?
Sim, no entanto você deve atentar ao fato de não incorporar material produzido por outrém cuja licença não é compatível com o projeto. Eventualmente, suas notam podem conter trechos ou exercícios copiados de livros cuja licença não é livre. Lembre também de adaptar seu material à folha de estilo do projeto.

\item Os organizadores do projeto garantem que o conteúdo esteja correto?
Não. Como em qualquer publicação, o material pode conter erros e imprecisões. No entanto, fazemos todo o esforço de revisão para manter a correção do texto e deixamos diversos canais de comunicação abertos para a remoção de erros. O material provido pelo REAMAT tem fins educacionais. Em caso de dúvidas, sempre consulte um professor qualificado.

\item A expressão ``conteúdo livre'' não é uma má tradução de ``free content'', a qual deveria ser vertida como ``conteúdo gratuito''?
Não, conteúdo livre e gratuito são conceitos distintos. Veja a definição de recursos educacionais abertos.

\item REA não é a mesma coisa que ensino à distância (EAD)?
Não, embora recursos educacionais abertos sejam frequentemente disponíveis online, eles podem ser utilizados em atividades presenciais.

\item Tenho outra pergunta, sugestão ou aviso. Que faço?
Considere entre em contato pelo nosso \href{https://www.ufrgs.br/reamat/forum.html}{fórum} ou escreva para \href{mailto:reamat@ufrgs.br}{reamat@ufrgs.br}.
\end{enumerate}
\end{document}
